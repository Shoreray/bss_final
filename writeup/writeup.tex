\documentclass[conference]{IEEEtran}

\usepackage{cite}
\usepackage{listings}

\begin{document}

\title{Finding Injection Vulnerabilities in Server-side SQL Libraries using Symbolic Execution}

\author{
	\IEEEauthorblockN{
	Kangqi Ni,
	Xiangyu Li
	}
	\IEEEauthorblockA{Georgia Institute of Technology, USA, \{kni3, xiangyu.li\}@cc.gatech.edu}
}

\maketitle

\begin{abstract}

SQL injection vulnerability used to be a prevalent security threat to web servers. Nowadays, it is most mitigated by the use of SQL libraries that pre-processes user inputs. Developers typically trust them to perform input sanitization. However, in case a SQL library does not do the input sanitization sufficiently, it could be a severe thread to all the web sites that uses the library. In this work, we proposed a technique that detects vulnerabilities in the SQL library code that could cause potential SQL injection of the application. We also conducted an empirical study to show the effectiveness of the technique.

\end{abstract}

\IEEEpeerreviewmaketitle

\section{Introduction}
\label{introduction}

SQL injection vulnerabilities used to be a prevalent, and serious security threat to web applications. Because of insufficient validation and sanitization of untrusted user input, attackers could trick the server side code to execute in unintended ways by using carefully constructed inputs, causing unauthorized access to sensitive data.

Most SQL injection attacks happen when the user inputs include strings. These string from user inputs are typically intended to be used only as string values, instead of SQL commands. However, if the string provided by the user could potentially break out of its string context, part of the user input would be used as SQL keywords, violating the intended semantic of the SQL query. 

Consider the following simple SQL query:
\begin{lstlisting}[breaklines=true]
SELECT * FROM login 
WHERE username = `name';
\end{lstlisting}
where `name' is an input from the user.
If the user specified name to be ``fake-user' OR 1 = 1;'', the query result would return everything in the login table regardless of username. This is because the single quotation mark in the user-specified string broke out of the string context of the SQL query, and thus the ``OR 1 = 1;'' changed the semantic of the where clause.
To prevent this, the server side need to perform appropriate sanitization of user inputs, escaping the single quotation character, which could break out of the string literal context.

Recently, the threat of SQL injection vulnerabilities is greatly mitigated because the widely use of higher-level SQL libraries instead of constructing the raw query statement by the application itself and sending it to the SQL engine directly. The SQL libraries are typically well written and tested in terms of sanitizing user inputs and preventing SQL injections. 

However, in case that the SQL library being used did not perform sufficient inputs sanitization and thus has SQL injection vulnerabilities, the consequence is severe. Application developers usually trust SQL library not to contain this kind of issues and will rely on the library code to perform anti-SQL-injection processing for them. Such vulnerabilities are unlikely to be discovered by application developers. Also, a SQL library is potentially used by large population of websites. The security threat has more impact than the security vulnerabilities in one particular website. Moreover, if the vulnerable SQL library is open-source, the attackers would have full knowledge of the bug and could devise sophisticated ways to exploit it. Thus, during the development of a SQL library, the developers of the library code wanted to make sure that there is absolutely no such vulnerabilities in the library code.

In this work, we present a technique that detects potential SQL injection vulnerabilities in SQL library code using symbolic execution. The technique works on the SQL-library-API level. In particular, it works on SQL-library APIs that produces valid SQL queries and checks whether the resulting SQL query could potentially expose injection vulnerability, for any arbitrary user inputs. Given an API method in the SQL library, the technique executes the API method symbolically, with the user inputs to the API method being marked as symbolic values. It tracks how the user inputs are used to produce the final SQL query and uses contraint solvers to determine whether there could be any user inputs that manifests a SQL injection attack. We implemented the technique in Python and the implementation works on Python programs.

We conducted an empirical study on the proposed techinque. The subjects we picked should 1) be purely in Python; 2) produces the full SQL query statement in an intermediate step or as the final output. According to these criteria, we picked a real-world SQL library, Python-SQL, as our subject. We also wrote a simple, small SQL library as a benchmark to assess the SQL-injection vulnerability detection capability of our technique.
As a note, initially, we hoped that this technique could be used to verify more widely-used SQL libraries such as sqlite on Python, and SQLAlchemy. However, after investigating these subjects, we found that these libraries does not produce the full SQL query. Instead, the logic of executing the SQL query is scattered in the code. Thus, in principle, our approach of finding SQL-injection vulnerabilities does not work on these SQL libraries.

The contribution of the work includes:

\begin{itemize}

\item The formulation of a symbolic-execution-based approach that detects SQL injection vulnerabilities in server-side SQL libraries.

\item An implementation of the approach that works on SQL libraries written in Python.

\item An empirical study that shows the effectiveness of the approach in discovering SQL-injection vulnerabilities.

\end{itemize}

\section{Our Approach and Implementation}
\label{approach}

\subsection{Overview}

Our proposed technique works on API method level. Given an API method of the SQL library, it verifies that no user input to the API method could possibly break out of its intended string context and be executed as SQL keywords.
First, the method is executed symbolically, with the input parameters marked as symbolic values. The symbolic execution engine explores as many program paths as possible in a breadth-first manner, until the resource limitation is reached. For each program path it explored, it records the symbolic expression of the resulting final SQL query statement, which is expressed in terms of the symbolic inputs, as well as the path condition from the starting of the symbolic execution to the point where the final SQL query is generated. The tool then uses the symbolic expression of the SQL query and the corresponding path condition of that particular program path to construct the condition, that there are some input being able to break out of its string context and executed as SQL keywords, which in turn indicates a potential SQL injection vulnerability. This condition is sent to the constraint solver to determine its satisfiability. If the condition is satisfiable, it indicates that the vulnerability is present. In this case, the tool also gives an example input that triggers this vulnerability.

We implemented this technique in Python. It consists of the symbolic execution engine, the SQL injection condition generator, and an off-the-shelf constraint solver.
The current implementation works for SQL libraries that are written purely in Python, but the approach is generalizable to other languages.

In the remaining part of this section, we'll talk about the implementation of each component of our implementation in details.

\subsection{Symbolic Execution}
\label{symbolic execution}

The symbolic execution engine is implemented in Python, without modification to the Python interpreter or runtime. Strictly, it is implemented as a concolic execution engine that evaluates the program both concretely and symbolically at the same time.

The concolic execution engine is implemented by extending primitive types in Python. The overloaded types, also referred to as concolic types, not only represents the concrete value of the original type, but also keeps track of the symbolic representation of its value, in terms of the symbolic inputs. The current implementation supports concolic int type and concolic string type, which are considered relavent to SQL injection vulnerabilities. Floating point numbers are not tracked symbolically because, in contrary to integer values, which could be used in string-indexing-related operations, they are unlikely to trigger injection vulnerabilities. Array index that depends on symbolic inputs is not tracked either because the string input from the user has to flow into the SQL query to trigger any potential injection vulnerability anyway.

The concolic types track the symbolic expression by adding a attribute to the instance of the original primitive Python type, that points to expression tree of the current concolic value. The concolic types override the common operations of the original primitive types. When the overridden operation is performed on the concolic value, it calculates the concrete result as well as the new symbolic expression associated with the result, which is typically also a concolic type value.

The concolic integer type supports all the built-in operations on integers. One special case is the comparison operation. Comparison of integer type values yields bool type values in Python. However, the bool type cannot be extended according to the language specification. We handle this by just adding its value to the constraint whenever we get a concolic bool value.

The concolic string type partially supports string operations in Python. If the operation is not supported by the concolic type specifically, the program will just treat the string value as an ordinary string and keeps executing. However, in this case, the symbolic information associated with the concolic string would be lost. The set of operations supported by the concolic string type is actually determined by the operations that can be solved by the string constraint solver we used. This is because, although we can represent all string operations easily, but the resulting symbolic expression would not be useful if it cannot be understood by the string constraint solver. Currently, we support string comparison, concatenation, getting length, contains, startswith, endswith, slicing, finding sub-string, stripping, splitting and replacing operations.

The concolic execution engine starts from a default set of value assignments of the symbolic variables. It evaluates the program concolically until the program finishes and collects the path condition. Then it systematically negates each clause in the path condition. For each new path condition generated in this way, it uses the constraint solver to determine whether it is satiefiable and comes up with an value assignments if it is. The concolic execution engine executes the program with the new value assignments and continue this process iteratively, until all program paths are explored or the resource limit is reached.

\subsection{SQL Injection Condition Construction}

For each program path during the symbolic execution of the SQL library API method, we extract the symbolic representation of the full SQL query constructed by the SQL library code as well as its path condition down to the point where the SQL query is constructed. 

For SQL libraries that output the final SQL qeury as a result, this is achieved by simple getting the symbolic expression of the final result and the whole path condition.
For SQL libraries that produces the SQL query as an intermediate result to be sent to the SQL execution engine, we used a hook function that traces the function calls of the Python program to get the symbolic expressions and path conditions. The user of the tool need to specify the function and parameter to retrieve as the final SQL query.

To avoid SQL injection vulnerabilities, the SQL library is expected to properly pre-process the strings from user inputs by escaping the single quotation characters with two consecutive single quotes. Thus, if the final SQL query could contain unescaped single quotation characters that is coming from user inputs, it is considered to be a potential SQL injection vulnerability.

We construct a constraint, that is solvable by the string solver, from the symbolic expression of the SQL query statement and the path condition. According to the aforementioned principle, the SQL injection condition is constructed in the following steps:

\begin{itemize}

\item Remove the single quotation literals from the expression tree that are concatenated to the SQL query string by the library code. This is done by walking through the symbolic expression tree of the final SQL query string, and removing all the single quotation literals that are connected to the expression tree by the concatenation operation. We remove these single quotation characters because they are added explicitly to retain the legitimate form of the SQL statement structure and are considered to be legal and safe.

\item Remove two consecutive single quotation marks in the final SQL query string. This is done by involing the symbolic version of the replace operation on the query string. These characters are escaped properly according to SQL syntax and are also considered safe.

\item At this point, if the SQL query string could still contains single quotes, these single quotes must be from user input strings. We assert that the final SQL query, after the previous two steps of pre-processing, still potentially contains single quotation marks. This constraints conjuncted by the path condition is the condition for a potential SQL injection vulnerability to exist. 

\end{itemize}

We send the SQL injection condition to the string solver to determine its satisfiability. If it is satisfiable, it indicates that there is a potential SQL injection vulnerability. In this case, the solver also generates an input to manifest this vulnerability.

\subsection{Notes of the Implementation}

The tool is implemented and tested in Python 2.7 and might not work on Python versions higher than 3.0.
The constraint solver we used is the Z3 solver from Microsoft. We also used the Z3-str extension to solve constraints that contains string operations.

The performance and functionality is mainly restricted by the constraint solver. Currently, the Z3-str string solver does not perform very well on the string replacing operations. First, there is no global replacing operator. We need to iteratively replacing the first occurrence until the string does not contain the target sub-string anymore. Second, for string constraints that includes multiple string replacing operations, it took excessively long time to come up with the solution.


\section{Evaluation}
\label{evaluation}
% discuss our own SQL library benchmark and several SQL libraries we have found
Our evaluation consists of two experiments. In the first experiment, we test our framework on python-sql 0.4~\cite{python-sql}. In the second experiment, we evaluate it on a small self-crafted SQL library to intentionally sweep common cases in practice. All experiments are testing internal logics of APIs exposed by the libraries, and run on an Intel Core i7-4712HQ machine with 16GB memory.

\subsection{Python-sql}
\label{python-sql}

\subsection{Crafted Python SQL Library}
\label{crafted-sql}

\section{Discussion}
\label{discussion}
In this section, we are discussing the two main limitations of our framework. One is the constraint solving, and the other is the failure to work on several exising python libraries.

\subsection{String Solver}
% discuss different solvers
Constraint solving plays an important role in web program analysis for the purpose of test generation for coverage, bug finding and vulnerability detection. The reason is that solver-based analysis tools enable more precise analysis with the ability to generate intersting bug-revealing inputs. Furthermore, solver-based analysis tools are often more robust and easier to build than otherwise. However, many string solvers such as HAMPI~\cite{hampi}, DPRLE~\cite{dprle}, and REX~\cite{rex} support only string operations. Such logics are not sufficient expressive for many program analysis since non-string operations are also widely used in SQL query checking. More importantly, the string and non-string operations interact in subtle ways leading to program errors that are hard for humans to find without automation. Finally, a string-only analysis will likely miss pure integer or string-to-integer constraints (e.g., length of string) thus resulting in path constraints that are not precise enough, leading to false positive. Therefore, we adopt Z3-str~\cite{z3-str}, most up-to-date satisfiability solver that supports a rich combined logic over string and non-string operations aimed at symbolic, static and dynamic analysis, as constraint solver in our framework.

As illustrated in~\ref{symbolic execution}, the capabilities of our framework are significantly constrained by how powerful the constraint solver is, in terms of the number of supported string operations and the complexity of solvable expressions. Due to the issue brought up in~\ref{evaluation}, our framework fails to exhibit its supposed effectiveness. However, the proposed approach holds promise in sync with the advance of techniques in string solvers.

\subsection{Python Libraries}
Many of the more mature databases support the concept of prepared statements. They can be thought of as a kind of compiled template for the SQL that a python SQL library wants to generate, that can be customized using variable parameters. Prepared statements have two major properties:
\begin{itemize}
\item The query only needs to be {\bf parsed once}, but can be executed multiple times with the same or different parameters. {\bf When the query is prepared, the database will analyze, compile and optimize its plan for executing the query}. By using a prepared statement the SQL library avoids repeating the analyze/compile/optimize cycle.
\item If a SQL library exclusively uses prepared statements, the developer can be sure that no SQL injection will ocurr (however, if other portions of the query are being built up with unescaped input, SQL injection is still possible).
\end{itemize}
The typical workflow of using a prepared statement is as follows:
\begin{itemize}
\item {\bf Prepare}:
The statement template is created by the SQL library and sent to the database management system (DBMS). Certain values are left unspecified, called \emph{parameters}, \emph{placeholders} or \emph{bind variables}.
\item {\bf Assemble}:
The DMBS parses, compiles, and performs query optimization on the statement template, and stores the result without executing it.
\item {\bf Execute}:
At a later time, the SQL Library supplies (or binds) values for the parameters, and the DBMS executes the statement (possibly returning a result). The SQL library may execute the statement as many times as it wants with different values.
\end{itemize}
Since our symbolic engine is to perform concolic execution on the internal library code, and capture the complete query before sent to the database, {\bf the design itself} sets up the scope of friendly python SQL libraries our framework supports. Unfortunately, both sqlite and SQLalchemy fall outside our scope, and thus are eliminated from our evaluation.

\section{Conclusion}
\label{conclusion}
We proposed a framework that detects potential SQL injection vulnerabilities in python SQL library using concolic execution. We encoded the SQL-injection condition as a set of constraints, and converted the problem to be a SAT problem determined by a constraint solver. Finally, we revealed future improvements on string solver and contributed a small SQL library benchmark for assessment on the effectiveness of proposed framework. In summary, we have made following contributions:
\begin{itemize}
\item We discovered certain internal mechanism widely adopted by some python SQL libraries, which prevents exposing complete SQL query at the library level.
\item We revealed existing limtations on state-of-art string solvers.
\item We designed our own SQL library benchmark to capture common cases in practice, thus thoroughly evaluating robustness of our framework.
\item We proposed a practical SQL-injection condition encoding scheme.
\end{itemize}

\section{Acknowlegements}
We would like to thank Taesoo Kim and Byoungyoung Lee for their help and comments. We also thank other classmates for their reviews.

\bibliography{reference}
\bibliographystyle{plain}

\end{document}
